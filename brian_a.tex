% \documentclass{article}
% \usepackage[utf8]{inputenc}
% \usepackage{hyperref}
% \usepackage[document]{ragged2e}
\documentclass[./ProjectReport.tex]{subfiles}
% \usepackage{indentfirst}

\begin{document}

% \maketitle

\begin{center}
\par
\href{https://www.biorxiv.org/content/10.1101/2020.05.29.124651v1}{https://www.biorxiv.org/content/10.1101/2020.05.29.124651v1}
\end{center}

\setlength{\parindent}{10ex}
This research article is published by Erie D. Boorman, a UC DAVIS psychology professor. He tells about how humans
 make decisions based on the relationship with the hippocampus(HC) and entorhinal cortex(EC), theoretically managing
 memory, time, and space rather than decision making. Through non-spatial tasks and 2-D cognition maps, it suggests
 that people make quick and accurate decisions with relative memory. Moreover, he proves that HC and EC are directly
 relative decision-making by proving the efficiency of decisions made in various memory situations and observing brain
 activity while making decisions.
\par
He assumes that HC and EC do not interact with each other. He then compares the correctness of decision people have
 relative memory with people do not have familiarity and found that memory significantly helps with decision making
 with p < 0.05, rejecting the null hypnosis saying is irrelevant. Also, he observes brain activity and proves that
 there is a significant effect that EC extends to HC with p < 0.05. However, using a confidence interval showing the
 level of interaction between EC and HC would be more convincing and informative. The bar charts presenting brain
 activity rate provide standard error that can help create its confidence interval. Plus, He uses the null hypothesis,
 assuming they are not relevant, but it’s very likely to be rejected the hypothesis is too extreme.

\end{document}
