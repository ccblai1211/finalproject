% \documentclass{article}
% \usepackage[utf8]{inputenc}
% \usepackage{hyperref}
\documentclass[./ProjectReport.tex]{subfiles}

\begin{document}

\maketitle

\begin{center}
\par
\href{https://static1.squarespace.com/static/504114b1e4b0b97fe5a520af/t/56f84581f850827e6f46c442/1459111298221/DuranteEastwickFinkelGangestadSimpson2016AESP.pdf}{Click here for paper}
\end{center}


\setlength{\parindent}{10ex}

The paper is focused on research of how people mate. Individuals in relationships are often determined to maintain a healthy relationship. On the other hand, research has also shown that individuals are motivated to explore their own sexual interests even if it comes at the expense of their relationship. In order to explore this topic, the article focused on the role of the ovulatory shifts in a woman and the effect this has on a relationship. The article indicated whether this would enhance the relationships women are in or  have them seek new mates. In conclusion, the study had found that women during their ovulation period will seek out male individuals that possess short-term genetic qualities as opposed to those having long-term characteristics.

\par
The paper found that the p-values are aligned with their hypothesis of women having a shift in mate preference near ovulation that depends on the mate characteristics. The distribution of p-values is right-skewed with p-values being less than 0.01 occurring more often than p-values being greater than 0.04. Although this is a strong indication of the hypothesis being correct p-values can always have false-positives. In this case, the metric for finding the p-values is based on personal preference of what women believe are the characteristics a mate possesses. Since this parameter is subjective it can lead to incorrect results.


\end{document}
