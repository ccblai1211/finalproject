\documentclass[./ProjectReport.tex]{subfiles}

% \usepackage[a4paper, total={7in, 10in}]{geometry}


\begin{document}
\maketitle
\begin{center}    
    The paper analyzed here is \href{https://www.cs.ucdavis.edu/~filkov/papers/icse2016focus.pdf}{https://www.cs.ucdavis.edu/~filkov/papers/icse2016focus.pdf}
\end{center}

The paper in question aims to analyze how Multitasking affects a programmers's
performance. Tools like GitHub have allowed programmers to work on multiple Projects
at ease, however, multitasking comes at a cognitive cost. Frequently switching projects 
and contexts can lead to distractions, sup-par work, and greater stress. 
This paper aims to analyze ecosystem-level data on a group of programmers
working on a large collection of projects and in turn develop models to measure
the rate and breadth of a developers’ context-switching behavior.
The paper manages to conclude that the most common reason for
multitasking is interrelationships and dependencies between
projects and that the rate of switching and breadth 
(number of projects) of a developer’s work matter.

To reach the conclusion, the paper uses p-values and the 95\% confidence 
intervals in order to test the hypothesis. However we know we cannot rely 
solely on them as that can lead to small P values even if the declared 
test hypothesis is correct, and can lead to large P values even if that 
hypothesis is incorrect. For instance, in the paper a p-value of 0.01 is achived
for most of the factors however, factors such as "Feeling more productive" is
a subjective matter and a p-value shoudn't be the only thing that decides
the success of a research study.
\end{document}