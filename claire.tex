\documentclass[./ProjectReport.tex]{subfiles}

\begin{document}

\begin{center}
    The paper analyzed here is https://www.frontiersin.org/articles/10.3389/fnins.2018.00379/full    
\end{center}

This paper analyzed the utility of the Magnetic Resonance Parkinson Index (MRPI) as a biomarker 
for Fragile X-associated tremor/ataxia syndrome (FXTAS), a neurodegenerative disorder. It was 
concluded that while MPRI may not be a useful biomarker for FXTAs, it was found that middle 
cerebellar peduncle (MCP) width, midbrain and pons cross-sectional area were reduced in patients 
with FXTAS when compared to both the premutation carriers without FXTAS and the controls. It was, 
however, also found that age was an important predictor of midbrain and pons cross-sectional area. 
Further, a subset of premutation carriers who later developed FXTAS symptoms had a reduced MCP width 
in their follow-up visit when compared to their initial visit. Thus, it was concluded that decreased 
MCP width may be one of the first notable signs of FXTAS, and thus a biomarker to identifying FXTAS 
at risk patients. 

This paper reached their conclusion using p-values to test their 
hypothesis. p-values determine whether the null hypothesis can be rejected. Significance testing 
assumes that the null hypothesis is true until it is proven otherwise. The smaller the p-value is, 
the more the null hypothesis can be rejected with greater certainty. The significance level for p < 
0.007 was set for all group and regression analysis, and the Bonferroni post-hoc analyses were set at 
p < 0.050. Using p-values for multiple variables can lead to p-hacking, where no trait has any real impact, 
but one appears significant due to sampling variation. However, this study mitigated this issue using the 
Bonferroni method. The Bonferroni method allowed the study to filter the other possible biomarkers to just 
MCP width. Lower p-values can sometimes be interpreted as proving that there is a stronger relationship 
between two variables where the relationship does not exist, as p-values is just an indicator for the likelihood 
of the null hypothesis being false. Confidence intervals, on the other hand, would provide a range in which 
the true value is within a certain probability. Confidence intervals would also allow the study to show the mean 
change of the traits (MCP, MRPI, midbrain and pons cross-sectional area)

\end{document}